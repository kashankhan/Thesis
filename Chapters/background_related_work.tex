
\chapter{Background and Related Work}

This chapter will establish the foundation of Persuasive recommendation system along with active learning and critiquing approach. Prior to in depth analysis, it will provide important background information along with some required definitions. Additionally, related work will presented, as the chapter proceed further to the end..\newline

\section{Definitions}

\subsection{Recommender System}

Recommender Systems (RS) are search tools, which supports user decision-making by providing the suggestion that, are according to their interests. Such systems are widely uses from social networking through e-commerce sites in order to achieve different purposed. In e-commerce site, they help not only to serve the customer by suggesting items according to their preferences but also support business to improve in its sale. On the other hand in social network site, to suggest friends or pages like according to user preferences. According to Ricci [Ricci, 2010] "Recommender Systems are information search tools that have been recently proposed to cope with the "information overload" problem, i.e., the typical state of a web user, of having too much information to make a decision". Proposed solution [Resnick and Varian, 1997] is an intelligent system that suggests the product or service that fulfill the user’s preference in given context or situation. Suggestions provided by such systems are depended on the model how they are keeping information. Majority of recommender systems are typically community based. In this kind of modeling suggestions are depend about item popularity among the user. Where popularity is calculated by ratings. Important question that arise in such systems are to find item accuracy according user preferences. On the other hand Personalized models are used that depends on the various factors which includes user’s preferences, history of bought/liked items, or the items the user has ranked in the past. Various techniques are use in the developing of recommender system. Classification of recommendation [Burke, 2007] will be discussed in next section.

\subsubsection{Content-based filtering}

\subsubsection{Collaborative-based Recommendations}

\subsubsection{Knowledge-based Recommendations}

\subsubsection{Hybrid Recommender Systems}


\subsection{Mobile Recommendations}

\subsection{Persuasive Recommedations}

\subsection{User Profiling}

\subsection{Food Profiling}

\subsection{Contexts}

\subsection{Conversation Critiquing and Active Learning}



\section{Related Work}

\subsection{User's Food Preference Extraction for Personalised Cooking Recipe Recommendation}

\subsection{Knowledge Base Framework for Development of Personalised Food Recommendation System}

\subsection{Interactive Explanations in Mobile Shopping Recommender Systems}

\subsection{Active Learning Strategies for Exploratory Mobile Recommender Systems Interactive Explanations in Mobile Shopping Recommender Systems}