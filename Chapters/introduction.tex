\chapter{Introduction}
\pagenumbering{arabic}%Ab hier, werden arabische Zahlen benutzt
\setcounter{page}{1}%Mit Abschnitt 1 beginnt die Seitennummerierung neu.
\thispagestyle{empty}

The motiviation behind this Master Thesis is to implement and evaluate a Healthy Food Recommendation System for Mobile. In the beginning, it provides overview and give the reasonsing about the selection of system. Relevant background and related work is presented, Followed by the development and design process together with the evaluation process will be presented. \newline

This chapter will enlighten the motivation behind develped system in Section 1.1. Followed by goals in Section 1.2. Whereas Section 1.3 will provide a breif outline on structure of thesis.

\section{Motivation}\label{motivation}

Rapid innovation and significant advancement in the field of technology and scientific research has made smartphone a primary computing and communication device. Smartphone is now become a necessity of life and people use it as an assistant for their day to day work. According to latest survey more than half of Internet traffic is accounted by mobile device. Enhancement in communication technology and flexible data option provided by network operators has increase relevance of interactive mobile applications. Packed with hundreds of features smartphone use different applications for variety of functionalities which required internet connectivity. Furthermore, smartphone support touch screen and rich support of multimedia and other application take the user experience to the next level.\newline

Suggestions are the important factor of our day-to-day life. From watching a movie, to cooking and to on shopping. We need valuable advice. Recommendations are always helpful in choosing a better alternative. It not only save time but also minimize the individual's effort.\newline

Recommender system are increasingly popular now are days. Form e-comers to movie websites, they not only help to increase business but also behave an personalize user preference assistance. Another aspect of describing Recommend system is filtering technology that user to filter suggests the information to user according to his taste. In order words we can also say that Recommender systems are the smart search engine, which suggest result by, compare different items with each other. Research and advancement are going on this domain in order to improve the quality of recommendations. \newline 

The most important goal for recommendation system designers is user willingness to peruse recommendation provided by the system. Fundamental process of recommendation is finding and conceptualizing relationship of item, current context and how the message is communicated, opens up the way of persuasiveness in recommendations.\newline

Services provided by recommendation system through e-commerce to cooking are numerous in nature. Searching of products returns an overwhelming set of options. For instance, Comparing and Filtering of products among irrelevant set and find the suitable product. Such techniques work fine with web interface, whereas, smartphones they are not very useful due to hardware limitations. Critique-based recommendation helps in revision and acquisition of user preference, in order to improve quality of recommendations.\newline

One the basic need is Food among human. Good health represents proper dietary habit. However, diet plan is always based on person’s physical conditions like gender, weight, age and health status. Furthermore, taste and food preference is differ among individuals. Therefore, creating balance dietary plan based on individual’s taste and health preference is always challenging.\newline

The World Health Organization [1] is predicting that the number of obese adults worldwide will reach 2.3 billion by 2015 and the issue is attracting increased attention. Therefore, electronic food management systems have become a hot topic and, are under consideration to replace traditional paper based program. Idea of using electronic devices for health related matter is not new; similar devices are in use by patients for medical reasons e.g. Glucometer, and blood pressure monitor. People want, to carry better life style and to live healthy. Therefore popularity of food monitoring systems is getting popular.  These systems are not only providing valuable services but hold user preferences and keeps history to provide more personalize recommendations. Recommendations are based on food ratings and browsing histories.\newline

Food recommendations have gotten a tremendous amount of success and still in research phase for further improvement. Along with significant advancement and feature set like similar recipes, recipe nutrition detail, where to buy ingredients from some research, some wholes are still remaining. Indeed recommendation techniques like collaborative, content and knowledge based filtering are good for job done. But food domain is not quite simple. User preference and taste not only change by their mood but much more depend upon their health. Therefore, Active learning and critiquing techniques are required to improve better recommendation. So that user can give their feed back and get what ever his preferences are. Mostly approaches are done critiquing by using rating of recipes and generate their result by using celebrative or content based filtering. Similarly, knowledge based filter digs some more; here rating is based on ingredients. Furthermore, persuasion of recommendation is always not guarantied in all cases. Clearly the system is not able to provide the best recommendations due to its detachment from the current situation; what is lacking in these approaches are intersection of persuasiveness, active learning and critiquing and last but not the least user preference context.\newline




For thesis we use iOS for develpment.

focus of  this work

\section{Goals}

\section{Outline}