\chapter{Introduction}
\pagenumbering{arabic}%Ab hier, werden arabische Zahlen benutzt
\setcounter{page}{1}%Mit Abschnitt 1 beginnt die Seitennummerierung neu.
\thispagestyle{empty}

The motiviation behind this Master Thesis is to implement and evaluate a Healthy Food Recommendation System for Mobile. In the beginning, it provides overview and give the reasonsing about the selection of system. Relevant background and related work is presented, Followed by the development and design process together with the evaluation process will be presented. \newline

This chapter will enlighten the motivation behind develped system in Section 1.1. Followed by goals in Section 1.2. Whereas Section 1.3 will provide a breif outline on structure of thesis.

\section{Motivation}\label{motivation}

Rapid innovation and significant advancement in the field of technology and scientific research has made smartphone a primary computing and communication device. Smartphone is now become a necessity of life and people use it as an assistant for their day to day work. According to latest survey more than half of Internet traffic is accounted by mobile device. Enhancement in communication technology and flexible data option provided by network operators has increase relevance of interactive mobile applications. Packed with hundreds of features smartphone use different applications for variety of functionalities which required internet connectivity. Futhermore, smartphone suport touch screen and rich support of multimedia and other applcaiton make the user experince to the next level.\newline

Suggesions are the important factor of our day to day life. From watching a movie, to cook a food we need valueable advise. Recommendation are always helpfull in chosing a better alternative. It not only save time but also minimize the individual's effort.\newline


Recommender systems are increasingly used in e-commerce websites, because they give businesses a strategic advantage over businesses without them. Recommender systems can be described as an information filtering technology that is used to present information on items to the user that are in line with the users tastes. These systems involve predictive models, heuristic search, data collection, user interaction and model maintenance. In other words we can see recommender systems as smart search engines which gather information on users and or items in order to give customized recommendations by comparing users and or items with each other.\newline

This smart search is achieved with help of historical data on products and customers. For instance the historical data can consist of ratings given by customers on products or another possibility is that previously purchased items are used. The historical data consists of information on items and customers. Businesses often have a large variety of products that are for sale. As the number of products increase customers will have more difficulty finding what they like. Recommender systems offer a solution as they are designed to provide customers with recommendations. As recommendations reduce the amount of time a customer needs to invest in searching for products of his or her liking, the overall user’s experience is positive.\newline


For thesis we use iOS for develpment.

focus of  this work

\section{Goals}

\section{Outline}