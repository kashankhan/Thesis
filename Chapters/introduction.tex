\chapter{Introduction}
\pagenumbering{arabic}%Ab hier, werden arabische Zahlen benutzt
\setcounter{page}{1}%Mit Abschnitt 1 beginnt die Seitennummerierung neu.
\thispagestyle{empty}

The motivation behind this master thesis is to implement and evaluate mobile-based food recommender system by using Persuasive and Active Learning techniques. In the beginning, it provides overview and gives the reasoning about the selection of system, presents the detail of design and development process presented and finally demonstrates the performance of two variants of the system against each other in a real user test. \newline

This chapter will enlighten the motivation behind developed system in Section 1.1.  What goals need to be achieved describes in Section 1.2. The last Section 1.3 will provide a brief outline on structure of thesis.

\section{Motivation}\label{motivation}

In this a mobile food recommender system developed, but why choose a mobile platform?. Rapid innovation and significant advancement in the field of technology and scientific research has made smartphone a primary computing and communication device. Smartphone is now become a necessity of life and people use it as an assistant for their day to day work. According to latest survey 60\% of internet traffic is accounted by mobile device \cite{shawnhessinger2014mobiletraffic}. Enhancement in communication technology and flexible data option provided by network operators has increase relevance of interactive mobile applications. Packed with hundreds of features smartphone use different applications for variety of functionalities which required internet connectivity. Furthermore, smartphone support touch screen and rich support of multimedia and other application take the user experience to the next level \cite{ricci2010mobile}.\newline

Next question arise why choose a food doamin?. The World Health Organization \cite{world2008information} is predicting that the number of obese adults worldwide will reach 2.3 billion by 2015 and the issue is attracting increased attention. Therefore, electronic food management systems have become a hot topic and, are under consideration to replace traditional paper based program. Idea of using electronic devices for health related matter is not new; similar devices are in use by patients for medical reasons e.g. Glucometer, and blood pressure monitor. People want, to carry better life style and to live healthy. Therefore popularity of food monitoring systems is getting popular.  These systems are not only providing valuable services but hold user preferences and keeps history to provide more personalize recommendations. Recommendations are based on food ratings and browsing histories.\newline

Food recommendations have gotten a tremendous amount of success and still in research phase for further improvement. Along with significant advancement and feature set like similar recipes, recipe nutrition detail, where to buy ingredients from some research, some wholes are still remaining. Indeed recommendation techniques like collaborative, content and knowledge based filtering are good for job done. But food domain is not quite simple. User preference and taste not only change by their mood but much more depend upon their health. Therefore, Active learning and critiquing techniques are required to improve better recommendation. So that user can give their feed back and get what ever his preferences are. Mostly approaches are done critiquing by using rating of recipes and generate their result by using celebrative or content based filtering. Similarly, knowledge based filter digs some more; here rating is based on ingredients. Furthermore, persuasion of recommendation is always not guarantied in all cases. Clearly the system is not able to provide the best recommendations due to its detachment from the current situation; what is lacking in these approaches are intersection of persuasiveness, active learning and critiquing and last but not the least user preference context.\newline

This work focuses on generation of food recommendation on a mobile platform. Recommendation provided the system should be persuasive in nature in terms of user interface and explanation of recommendation. Where explanation helps user to understand what factors that took into account while making a recommendation. Also allows her to correct the wrong assumption about her preference by the mean of Active Leering and critiquing. Generation of recommendations should rely on user context and her preferences. 
\newline

The following short description of the target scenario will illustrate the driving idea behind this research project.\newline

 \textit{Amy comes form gym and aiming to cook. However she doesn’t want spend her time by searching recipe from different recipe website.  She opens up the mobile app that display her recipe. Within few interactions she got some recipe that are according to her deity need and cooking time, along with the explanation why this recipe recommended to her. Feeling confident with the recommendation she cooks that recipe.}\newline

\section{Goals}

On the basis of scenario, describes in last section, this work reflects the goals which are stated below:\newline

A recommendation is valuable if it interests the user. To determine the generated recommendation is according to user interest entails to our first primary goal, which is offering Persuasive recommendations. Major factors should be considered before given suggestion is Message and Source. Therefore our Second goal is to implement Active Learning and Critiquing approach to justify our suggestion. Since Critiquing relies on context that’s why it is important to understand the Consumption and Accessibility context which infers our third goals.  Similarly understanding the food ontology refers in scenario helps us to understand forth goal of system. Finally how the user will interact with his device conveys our last goals, which is Mobile user interface.\newline

To achieve primary goals there are several other interesting secondary goals, which facilitate, how our primary goals should be achieved. Starting with the research phase, which includes question and answers to user how they want to use such system in order to achieve better usability. Next focus on existing search work how the other system implements food recommendation scenarios, finding out what are their weakness and strengths. Food ontologies understanding how they are interrelate with other. What factor in which recipes are dependent on in order to develop strong system. Understanding user context which time he prefers which recipe. Furthermore, it is important to research on what researches and related work are out there under Persuasive and Active learning and Critiquing system to grab the understanding, how we can get inspiration from their valuable approaches and work. Finally focus on user experience of such application is one of challenging task, how and where to show the important aspects of recipe in our interface, so that it is easy to learn and has improved usability in comparison with current market applications.\newline
Once the research phase has done next step to collect the functional and non-functional requirement of the system, which is collected by interviewing friends and family voluntaries. Once the system is build it has been tested with gathered functional and non-functional requirement and find out the limitation or boundary conditions of system. More over iOS client needs to be test with given requirement additionally user satisfaction should be required for usability test.\newline

Finally, evaluation of developed prototype by user study. In order to clarified the methodologies and processes followed by our selected approaches. After finishing the evaluation reflected results leads to potential improvements and opens up the new direction of research.\newline

\section{Outline}

Division of this thesis is split up into six chapters. \textit{Chapter1} contains introduced the ideas, motivations and goals.\newline

\textit{Chapter2} starts with background in which some definitions and classification of recommendation systems, Followed by different types of profiling and contexts that impact on recommendations. Furthermore, in related work section, pervious work of Persuasiveness, Critiquing and Personalized food recommendation techniques have been discussed.\newline

\textit{Chapter3} explains the Profiling and Context in details along with factors of recommendations. Moreover it covers algorithms that are used to develop the system.\newline

\textit{Chapter4} discuss the System design and architecture phase, which hold the all ERD, components view, servers on which system depend. In the end of the chapter API calls are mentioned which are provided by server.\newline

\textit{Chapter5} elaborates how the user study has been conducted by mentioning the goals, methods, and testing framework along with the dataset. In the end of this chapter measured results and discussion is mention.\newline
  
\textit{Chapter6} will summarizes the achievements and gives clues about further development and research.\newline
