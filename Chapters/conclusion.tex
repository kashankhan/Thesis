\chapter{Evaluation And Conclusion}

Despite the fact, how good the prototype in terms of algorithmic and design approaches. There will be some probability that elaborates, users are doing what they are not expected to do.  This leads to other features(s) that needs to be develop in order to improves the user’s satisfaction and increase his willingness to user the system. Therefore it is important that developed system should go through an evaluation proves before it goes live. Focus of evolutions to ensure that product is appropriate and the involvement of user through the design process. This chapter depicts the evaluation of prototype in a real user study and present the result.

\section{Motivation and Goals}

Motivation behind performing evaluation to determine, whether the process of recipe according to user interest in mobile critique-based recommender system can be improved by applying Persuasive Principles. As discussed earlier, purpose of developed Food Recommender System aim to be used in a real world situation. This established some aspects of the user study, including the development of a variant of the proposed application and assessment of the system.\newline

During the study, two variants of the system will need to be tested, one begin the basic interface design, without explanation and works on basic recommender system by providing star rating to the recipes. Whereas, the second system is the main output of thesis, with better infrastructure of recipes, sleeker interface and having explanation about the recipes. Additionally recommender system algorithm supports critiquing on both ingredients and recipes.\newline 

The study is designed in a way that each user has to test both variants of the application, which system is more appealing for the users. Focusing on real effect of recommendation depends on factors like user intent, context, way to present recommendations set and other. Thus experiment needs to provide evidence as the true value of evaluation.\cite{shani2011evaluating}. Additionally a single irritation of user study should not exceed with more then 20 min to maintain user interest. Considering the fact that user have to test two variant of application. However it is possible that user might have get some less qualitative results which is not due the fault from the system but because users were overwhelmed with long sessions.  \newline

The focus of evaluation was to measure the effect of persuasion by providing the recommendation in form of recipes. Also how active learning also system to change itself according to user preferences. However it exclude the non-relevant part like ingredients integration to improve the recipes.

\section{Data set Generation}

Data sets are necessary to create a pragmatic setup to represent a real world objects. Therefore, a data crawler needs to be developed as an open-source project that will crawl the recipes form any recipe data bank. Crawler developed by us is written in java.  Crawling recipe form data bank is a two steps process.  First fetches the recipes from databank on behave of food type and course.  In second step it will sync all those recipes whose details are not present in our database. Additionally recipes images are not been stored in our system instead of saving image we stores their URLs. Extracted data provide international recipes of different cousins. However, it provides functionally to add more different data source that provides recipes. To keep the amount of work reasonable items were associated with the following:

\begin{enumerate}
	\item Unique identifier for recipes
	\item 11 types of courses (e.g,. Appetizers, Bread)
	\item 91 Cuisine (e.g. American, Thai, Beverages)
	\item Images links of different sizes
	\item Preparation
	\item List of ingredients
	\item Popularity of recipe
\end{enumerate}

Currently our data bank has 1303 of recipes, 10037 ingredients. Furthermore, it can grow more depends on crawling method.

\section{Setup}

This section leads our discussion toward the selection of test hardware, different variants of application and testing framework for the sake of performing evaluation.

\subsection{Test Hardware}

Recommended hardware must have at least 320x 480 resolution and above, running iOS version 8.0 or later. In order words required device to run application is iPhone 4S and above.  

\subsection{Variants}