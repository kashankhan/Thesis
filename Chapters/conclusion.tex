\chapter{Conclusions and Future Work}


This thesis investigated, the development and impact of Persuasion along with Active-Learning critique-based mobile recommender system in lifestyle domain. The developed concept proposed the factors of Persuasion Principle, generation of explanation not only to make the system transparent but also have an impact of persuasion in it and also provide user controller in the process of recommendation.\newline

The system distinguishes user context and recommendation explanations. Where user preferences and context tells the system about current user interest and explanation justifies recommendation item relevance. For the getting the user context cooking time and recipe course was consider. While explanation part considers the user current context, vector of weighted recipe and liked and disliked ingredients. To make recommendation more persuasive health factor was introduced in explanation which contains the information about calories that recipes contains and how much user have to perform work out to burn such calories.  Finally the recommendation are made more interactive and follows the user-centric design and systems provide user option to correct the wrong assumption made by the system.\newline

An iOS mobile application was develop to measure the applicability of concept. Which should followed the user-centric design approach, having consisting UI like other market product. Prototype was presented to 32 real for evaluation. The purposed concept was able to achieve significantly improved result in terms of persuasion, explanation and critiquing on recommend items. In general users of the application appreciated the interface design and concept. Additional to this user are satisfied and want to cook the recommended recipe. In short proposed system was highly acceptable by the system and people want to have such application in market.\newline

\section{Future Work}
Future development may include the creation of more complex recommendation scenarios to test the capabilities of proposed concept. Users are not only allowed to add recipes into the system but also add different cooking method of same recipe by trying our different ingredients. In addition to this more health factor need to be consider in the generation of recommendation like BMI (Body Mass Index), current health status (like diseases and allergy that use may have) and integration of iOS health app so that user can track his health status. To make recommendation more effective system should be consider Healthy Eating Index and nutrition information of each ingredient in recipe.

On algorithmic side research might be to add machine-learning algorithm that would help system to generate more accurate results that would be more effective to the user and produce more diversify results. Which the help of this it would be more easy for the system to generate weekly dietary plan for user and recommend him. User can add his wishlist regarding recipes and system by looking his profile recommend him recipes. Furthermore, if user have some food restriction due to his health system would be recommend his alternative to that recipe ingredients and present to user if it is good for him.\newline

Other improvement might be to make the system pro-active by considering more contexts like location. Suppose on the ways to home form gym system recommend user some drinks that may be important to his health. Last important improvement be to test the application in real world scenario which more users and system is developed on cloud and make the system distributed to measure the effectiveness of developed system in terms of time and cost. 