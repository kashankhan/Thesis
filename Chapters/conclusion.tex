\chapter{Evaluation And Conclusion}

Despite the fact, how good the prototype in terms of algorithmic and design approaches. There will be some probability that elaborates, users are doing what they are not expected to do.  This leads to other features(s) that needs to be develop in order to improves the user’s satisfaction and increase his willingness to user the system. Therefore it is important that developed system should go through an evaluation proves before it goes live. Focus of evolutions to ensure that product is appropriate and the involvement of user through the design process. This chapter depicts the evaluation of prototype in a real user study and present the result.

\section{Motivation and Goals}

Motivation behind performing evaluation to determine, whether the process of recipe according to user interest in mobile critique-based recommender system can be improved by applying Persuasive Principles. As discussed earlier, purpose of developed Food Recommender System aim to be used in a real world situation. This established some aspects of the user study, including the development of a variant of the proposed application and assessment of the system.\newline

During the study, two variants of the system will need to be tested, one begin the basic interface design, without explanation and works on basic recommender system by providing star rating to the recipes. Whereas, the second system is the main output of thesis, with better infrastructure of recipes, sleeker interface and having explanation about the recipes. Additionally recommender system algorithm supports critiquing on both ingredients and recipes.\newline 

The study is designed in a way that each user has to test both variants of the application, which system is more appealing for the users. Focusing on real effect of recommendation depends on factors like user intent, context, way to present recommendations set and other. Thus experiment needs to provide evidence as the true value of evaluation.\cite{shani2011evaluating} \newline

