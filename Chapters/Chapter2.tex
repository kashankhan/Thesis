
\chapter{Theoretical Background}
\section{Important Foundations}
\subsection{Software Architecture}
During the last few years, complexity in modern software has increased dramatically. With this increased complexity and distributed nature of software, the power and importance of Software Architecture can be realized. Software Architecture is an Art, an art of structuring the software in terms of modules, components, services and layers, that makes it scalable, adaptable, maintainable and change tolerant.\newline

\textit{Architecture is defined by the recommended practice as the fundamental organization of a system, embodied in its components, their relationships to each other and the environment, and the principles governing its design and evolution} \cite{architecture_def1} \newline 

Software Architecture depicts the high level or abstract picture of the software. It makes lot easier for the architects to make design decisions, and identify critical issues at an earlier stage. A good  architecture has a well organized structure, which reduces the complexity and leads to better understanding of the system. Once we have achieved it, we can fulfill all the requirements.\newline

\textit{The software architecture of a program or computing system is the structure or structures of the system, which comprise software elements, the externally visible properties of those elements, and the relationships among them.} \cite{architecture_def2}

\subsection{Software Architecture Conformance Analysis}
The quality attributes such as performance, security, modifiability and reliability depend heavily on the architecture of the software. Quality is achieved if the implementation conforms to the constraints and design principles of the architecture. But unfortunately, the implementation never realizes the abstractions of the architecture completely. Often, there is a sizable gap between these two. This divergence can be due: 
\begin{itemize}
\item Negligence of the developer 
\item Lack of documentation for the design artifacts
\item Architecture flaw identified during implementation 
\item Natural evolution of the model/code  
\end{itemize}

Architecture conformance analysis includes processes and strategies, whose intent is to remove or reduce the level of divergence and keep the architecture sync with the implementation or vice versa.

\subsection{Architecture Violation}
\subsection{Software Dependency}
Software Dependency is a system level concept for measuring the independence of the system components. Increase in dependencies results in increased coupling between the system components which exhibits more defects than lower dependencies.\newline

Dependency is a relational concept between two entities (System, Components, Modules, Classes or Functions). This relationship can either be Functional Dependency i.e. function A calls function B or it can be Data-Related Dependency i.e. a data structure is being utilized in one function and modified in another.\newline

\begin{itemize}
\item Compile and Run-time dependencies 
\item Visible and Hidden Dependencies
\item Direct and Indirect Dependencies 
\item Local and Context Dependencies  
\end{itemize}

\subsection{Model-Code Gap}

\section{Related Work}